\clearpage

\begin{table}[h!]
\section{Reihen}

\begin{center}

% % % % % % % % % % % % % % % % % %
%Grundlegendes
% % % % % % % % % % % % % % % % % %	
\begin{tabularx}{540pt}{|p{100pt}|X|}
\hline
\rowcolor{Gray}
\multicolumn{2}{|c|}{\textbf{Grundlegendes}}\\
\hline

	Reihe & 
	Folge $\langle a_n \rangle = a_1,a_2...a_n \qquad$
	Folge $\langle s_1\rangle = a_1$ und $\langle s_2 \rangle = a_1+a_2$\newline
	Eine Reihe ist eine Folge ihrer Partialsummen $\lim\limits_{n\to\infty} s_n  =  \lim\limits_{n\to\infty}\sum\limits_{k=1}^{n}a_k = \sum\limits_{k=1}^{\infty}a_k = s$\\
\hline
	Konvergenz/Divergenz &
	Konvergiert die unendliche Reihe $\langle s_n\rangle $so besitzt sie die Summe s. $\qquad
	s=\sum\limits_{k=1}^{\infty} a_k$\newline
	Existiert der Grenzwert nicht, so ist die Reihe divergent.\newline
	Wenn man in einer Reihe endlich viele Summanden hinzu/weglässt, so bleibt sie Konvergent oder Divergent(nicht so bei Folge)\\
\hline
	Vertauschen \newline der Summanden &
	Für \textbf{unendliche Reihen} gilt , dass die einzelnen Summen untereinander {\color{red}nicht} vertauscht werden können\\
\hline
	Es gilt ausserdem&
	$a=\sum\limits_{k=1}^{\infty} a_k \quad b=\sum\limits_{k=1}^{\infty} b_k$ sind konvergente Reihen $\quad a_k \leq b_k\quad \forall n\in \mathbb{N} \qquad $dann ist$\quad {\color{red}a\leq b}$\\
	\hline
\end{tabularx}

% % % % % % % % % % % % % % % % % %
%Konvergenzkriterien
% % % % % % % % % % % % % % % % % %	
\begin{tabularx}{540pt}{|p{100pt}|X|}
\hline
\rowcolor{Gray}
\multicolumn{2}{|c|}{\textbf{Konvergenzkriterien}}\\
\hline

	Notwendiges\newline Konvergenzkriterium  &
	Wenn ein Grenzwert konvergieren soll,muss$\lim\limits_{n\to\infty}a_n = 0$ sein.\newline
	( Reihe kann trotzdem divergieren z.B. unbestimmt)\\
\hline
	Notwendiges\newline Divergenzkriterium&
	$\lim\limits_{n\to\infty}a_n \neq 0$\\
\hline
	Cauchyches Konvergenzkrit.&
	Es existiert ein $\epsilon > 0 \qquad \epsilon \geq s_0=\sum\limits_{k=1}^{n_0}$\qquad
	Nun gilt für alle $m>n>n_0 \qquad \vert \sum\limits_{k=n}^{m}a_k\vert <\epsilon$\newline
	Dann Konvergiert die Reihe, ansonsten divergiert sie. $(|s_m-s_n|< \epsilon)$	 \\
\hline
	Reziprokkrit&
	$ s = \sum\limits_{n=1}^{\infty} \dfrac{1}{n^\alpha} $
	$\begin{cases}
	$konvergent für$ & \alpha > 1\\
	$divergent für $ & \alpha \leq 1
	\end{cases}$\\
\hline
	Majorantenkrit  &
	Ist die Reihe $ \sum\limits_{n=1}^{\infty} c_n $ konvergent, so konvergiert auch die Reihe $ \sum\limits_{n=1}^{\infty} |a_n|$ und somit auch
	$\sum\limits_{n=1}^{\infty} a_n$ für $|a_n| \leq c_n$ (absolut).
	Dies gilt auch für $|a_n| \leq c_n$ erst ab einer Stelle $n_0 \in \mathbb{N}$.\newline
	\\
\hline
	Minorantenkrit.  &
	Ist die Reihe $ \sum\limits_{n=1}^{\infty} d_n $ gegen $+\infty$ divergent, so gilt dies auch für die Reihe $ \sum\limits_{n=1}^{\infty} a_n $ 
	bei $a_n \geq d_n$. \newline Dies gilt auch für $a_n \geq d_n$ erst ab einer Stelle $n_0 \in \mathbb{N}$. \\
\hline
	Quotientenkrit.&
	$ \lim\limits_{n \to \infty} \left|\dfrac{a_{n+1}}{a_n}\right| = \alpha $ der Reihe $ \sum\limits_{n=1}^{\infty}a_n \qquad$ \\
	
	Wurzelkrit.  &
	$\lim\limits_{n \to \infty} \sqrt[n]{\left|a_n\right|} = \alpha $ der Reihe $ \sum\limits_{n=1}^{\infty} a_n\qquad$
		$\begin{cases}
		\alpha < 1 & $(aboslut) konvergent$\\
		\alpha = 1 & $keine Aussage!$\\
		\alpha > 1 & $divergent$
		\end{cases}$ \\
\hline
	Integralkrit.&
	$\int\limits_{1}^{\infty}f(x)dx$ konvergent $\Leftrightarrow \sum\limits_{n=1}^{\infty}f(n)$ konvergent. \\
			
			&Gilt nur, wenn $f$ auf $ [1, \infty) $ definiert und monoton fallend ($f'(x) \leq 0$) ist. \\
			&Zudem muss $ f(x) \geq 0 $ für alle $x \in [1, \infty)$ sein.\\
\hline
\end{tabularx}


\end{center}
\end{table}	

% % % % % % % % % % % % % % % % % %
%neue Seite
% % % % % % % % % % % % % % % % % %	
\begin{table}[h!]
\begin{center}

\begin{tabularx}{540pt}{|p{100pt}|X|}

\hline
	Leibniz Krit. &
	Die \textbf{alternierende} Reihe $ \sum\limits_{n=1}^{\infty} a_n $ ist konvergent, wenn die Folge $\langle\left|a_n\right|\rangle$ eine monoton fallende Nullfolge ($\lim\limits_{n \to \infty}
	\left|a_n\right| = 0 $) ist. Monotonie mittels Verhältnis $\left( \left|\frac{a_{n+1}}{a_n}\right| \right)$, Differenz ($ |a_{n+1}| - |a_n| $) oder vollständiger Induktion beweisen.\\
	&
	Abschätzung Restglied einer alternierenden konvergenten Reihe: $|R_n|=|s-s_n|\leq|a_n+1|$\\
\hline
	Absolute Konvergenz&
	Eine Reihe $\sum\limits_{n=1}^{\infty}a_n$ heisst \textbf{absolut konvergent}, wenn die
	Reihe $\sum\limits_{n=1}^{\infty}|a_n|$ konvergent ist.\\
\hline
	Unbedingt\newline Konvergent & 
	Unbedingt Konvergent ist eine Reihe die durch umordnen einen anderen Grenzwert hat oder wird divergiert.\\
\hline
	Bedingt Konvergent &
	Unbedingt kann man umordnen, ohne dass sich konvergenz oder Grenzwert ändert.\\
\hline
\end{tabularx}

% % % % % % % % % % % % % % % % % %
%Potenzreihen
% % % % % % % % % % % % % % % % % %	
\begin{tabularx}{540pt}{|p{100pt}|X|}
\hline
\rowcolor{Gray}
\multicolumn{2}{|c|}{\textbf{Potenzreihen}}\\
\hline

	Grundlegend&
	$\sum\limits_{n=0}^{\infty}a_n(x-x_0)^n$ ist eine Potenzreihe mit Entwicklungspunkt $x_0$ und $a_n$ als Koeffizienten\\
\hline	
	Konvergenzkrit&
	$\sum\limits_{n=0}^{\infty}a_n x^n$ Es sei $\lim\limits_{n\to\infty}\sqrt[n]{|a_n|}=a
	\begin{cases}
	a=0 & \text{absolut Konvergent } \forall x\in\mathbb{R}\\
	a>0 & \text{absolut Konvergent für }|x|<\dfrac{1}{a} 
	\end{cases}$\\
\hline
	Konvergenzradius Wurzelkrit.&
	
	Ist die Folge $\langle \sqrt[n]{|a_n|} \rangle$ konvergent, so heisst die Zahl $\rho=\dfrac{1}{a}$ Konvergenzradius Wurzelkrit.\\
	Quotientenkrit.&
	$\rho=\lim\limits_{n\to\infty}\left\lvert\dfrac{a_n}{a_{n+1}}\right\lvert$\\
	
	Mehrere Summen&
	$\sum\limits_{n=0}^{\infty}a_nx^n$ hat $\rho_1 \quad \sum\limits_{n=0}^{\infty}b_nx^n$ hat $\rho_2 \qquad
	\rho = min\{\rho_1, \rho_2\} \quad $Dann gilt:\newline
	$\sum\limits_{n=0}^{\infty}a_nx^n+\sum\limits_{n=0}^{\infty}b_nx^n = \sum\limits_{n=0}^{\infty}(a_n+b_n)x^n \newline 
	\left(\sum\limits_{n=0}^{\infty}a_nx^n\right)\cdot\left(\sum\limits_{n=0}^{\infty}b_nx^n\right)= \sum\limits_{n=0}^{\infty}\left(\sum\limits_{k=0}^{n}a_kb_{n-k}\right)x^n$\\
\hline
	Ableitung Potreihen&
	$\left(\sum\limits_{n=0}^{\infty}a_nx^n\right)'=\sum\limits_{n=1}^{\infty}n\cdot a_nx^{n-1}\quad$ Der Konvergenzradius $\rho$ bleibt gleich\newline
	Es gilt auch: $f^{(i)}(x)=\sum\limits_{n=i}^{\infty}n(n-1)\ldots(n-i+1)\cdot a_nx^{n-i}$\\
\hline

	Aufleitung Potreihen&
	$\int\sum\limits_{n=0}^{\infty}a_nx^ndx=\sum\limits_{n=0}^{\infty}a_n\int x^ndx=\sum\limits_{n=0}^{\infty}\dfrac{a_n}{n+1}x^{n+1}$\\
\hline
	Taylor-Reihe&
	Für eine beliebig oft differenzierbare Funktion gibt es die Taylorreihe $\sum\limits_{n=0}^{\infty}\dfrac{f^{(n)}(x_0)}{n!}\cdot (x-x_0)^n$\newline
	Für alle Glieder der Taylorreihe muss die folgende Bedingung erfüllt sein $\lim\limits_{n\to 0} T(\xi)=0 $\\
\hline
\end{tabularx}	


\end{center}
\end{table}	

\begin{table}[h!]
\begin{center}
% % % % % % % % % % % % % % % % % %
%Grenzwerte
% % % % % % % % % % % % % % % % % %	
\begin{tabularx}{540pt}{|p{115pt}|p{140pt}|p{90pt}|X|}
\hline
\rowcolor{Gray}
\multicolumn{4}{|c|}{\textbf{Grenzwerte}}\\
\hline
		$\lim\limits_{n\to\infty}(1+\frac{x}{n})^n = e^x$ &		$\lim\limits_{n\to\infty}(\sqrt[n]{n^a}) = 1$ ($a$ const.)&
		$\lim\limits_{n\to\infty}(\sqrt[n]{n}) = 1$ &
		$\lim\limits_{n\to\infty}(\sqrt[n]{a}) = 1$ \newline($a > 0$ und const.) \\
\hline
		$\lim\limits_{n\to\infty}(\frac{K}{n!}) = 0$ ($K$ const.) &
		$\lim\limits_{n\to\infty}(\sqrt[n]{|p(n)|}) = 1$ ($p(n) \neq 0$) &
		$\lim\limits_{n\to\infty}(\sqrt[n]{n!}) = +\infty$ &
		$\lim\limits_{n\to\infty}(\sqrt[n]{\frac{K^n}{n!}}) = 0$ \newline($K > 0$ und const.)\\
\hline
		$\lim\limits_{n\to\infty}(\frac{n}{\sqrt[n]{n!}}) = e$ &&&\\
\hline		

\hline
\end{tabularx}	


% % % % % % % % % % % % % % % % % %
%Einige Reihen
% % % % % % % % % % % % % % % % % %	
\begin{tabularx}{540pt}{|p{180pt}|p{180pt}|X|}
\hline
\rowcolor{Gray}
\multicolumn{3}{|c|}{\textbf{Einige Reihen}}\\
\hline
	Geometrische:\newline
	 $s_n=\sum\limits_{k=0}^{n}a_0 q^{k}=a_0\cdot\dfrac{1-q^n}{1-q}$\newline
	 $s=\sum\limits_{k=0}^{\infty}a_0 q^{k}=\dfrac{a_0}{1-q}$&
	Arithmetische:\newline
	$s_n=\sum\limits_{k=0}^{n}a_0 +k\cdot d=\frac{n}{2}(a_1+a_n)$&
	Harmonische: (divergiert)\newline
	$s_n= \sum\limits_{k=1}^{n}\dfrac{1}{k}=1+\frac{1}{2}+\frac{1}{3}...$\\
\hline
	$\sum\limits_{n=1}^{\infty}\dfrac{x^n}{n^\alpha}: \rho = 1$ \newline
	für $\rho = 1$
	$\begin{cases}
	\alpha = 0 & divergent\\
	0 < \alpha \leq 1 & beachte \quad x\\
	\alpha >1 & konvergiert
	
	\end{cases}$&

	$\sum\limits_{n=0}^{\infty}\dfrac{x^n}{n!} \qquad
	\rho = +\infty\quad$ quotkrit.&
	
	$\sum\limits_{n=0}^{\infty}\binom{\alpha}{n}\cdot x^n = (1+x)^\alpha\quad$
	$|\rho| = 1$\newline
	p.m. $\binom{u}{k}= \frac{u!}{(u-k)!k!}$ \\
\hline
	
	$\sum\limits_{n=1}^{\infty}2^n(x-3)^n\quad |p|=\frac{1}{2} \newline [3-\rho,3+\rho]=[\frac{5}{2},\frac{7}{2}]$&
	$\sum\limits_{n=0}^{\infty}\dfrac{x^n}{n!}=e^x$&
	$\sum\limits_{n=1}^{\infty}\dfrac{1}{n^2}$\\
\hline
\end{tabularx}	

\end{center}
\end{table}	