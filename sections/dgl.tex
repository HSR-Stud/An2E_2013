\clearpage

\begin{table}[h!]
\section{Differentialgleichungen}

\begin{center}

% % % % % % % % % % % % % % % % % %
%Grundlegendes
% % % % % % % % % % % % % % % % % %	
\begin{tabularx}{540pt}{|p{100pt}|X|}
\hline
\rowcolor{Gray}
\multicolumn{2}{|c|}{\textbf{Grundlegendes}}\\
\hline
	Grundsätzlich&
	Eine Gleichung zur Bestimmung einer Funktion heisst Differentialgleichung, wenn sie mindestens eine Ableitung der gesuchten Funktion enthält \\
\hline
	Ordnung&
	Die Ordnung wird bestimmt durch die höchste Ableitung der gesuchten Funktion\\
\hline
	
	Anfangswertproblem&
	Funktion: $y^{(n)}=f(x,y,y',...,y^{(n-1)})$\newline
	Das Anfangswertproblem hat die Aufgabe, eine Funktion zu finden, die folgendes erfüllt: $\quad y(x_0)=y_0 \quad y'(x_0)=y_1\quad ...\quad y^{n-1}(x_0)=y_{n-1}$\newline
	Anfangswerte: $y_0, y_1,...,y_{n-1} \qquad$ mit Anfangspunkt $x_0$\\
\hline
	Existenz/Eindeutigkeit\newline
	(Piccard-Lindelöf)&
	Die Funktion $f(x, u, u_1, ..., u_{n-1})$ sei in einer Umgebung der Stelle \newline
	$(x_0, y_0, y_1, ..., y_{n-1}) \in \mathbf{R^{n+1}}$ stetig und besitzt dort stetige partielle Ableitungen
	nach $u, u_1, ..., u_{n-1}$ dann existiert in einer geeigneten Umgebung des Anfangspunktes $x_0$ genau eine Lösung des Anfangswertproblems\newline
	$y^{(n)} = f(x, y, y', ...,y^{(n-1)})$ mit $y(x_0) = y_0, y'(x_0) = y_1, ..., y^{(n-1)}(x_0) = y_{n-1}$ \newline
	$\frac{\partial f}{\partial y}$ ... $\frac{\partial f}{\partial f^{(n-1)}}\qquad$ endlich beschränkt $\Rightarrow$ eindeutige Lösbarkeit\\
\cline{1-2}
	&$y'=-\dfrac{x}{2}-\sqrt{4+\frac{x^2}{4}} \quad AW:y(0)=1$\newline
	$y'=f(x,y)\quad \dfrac{\partial f}{\partial y}=\frac{-1}{2\sqrt{y+\frac{x^2}{4}}} \qquad$
	Allgemein: $y\neq -\frac{x^2}{4} \quad$\newline
	für dieses AW-Problem AW einsetzen:$-\frac{1}{2\sqrt{1+0}}=\frac{1}{2}\Rightarrow$ eindeutig lösbar\\
\cline{1-2}
	&Anfangsbedingungen müssen ungabhängig sein: $y_0= ae^{x_0}+be^{-x_0}\quad y_1=ae^{x_0}-be^{-x_0} \Rightarrow det
	\begin{pmatrix}e^{x_0}& e^{-x_0}\\e^{x_0}& -e^{-x_0}\end{pmatrix}= -2\neq 0$\\
\hline
	
\end{tabularx}	


% % % % % % % % % % % % % % % % % %
%DGL 1. Ordnung
% % % % % % % % % % % % % % % % % %	
\begin{tabularx}{540pt}{|p{100pt}|p{100pt}|X|}
\hline
\rowcolor{Gray}
\multicolumn{3}{|c|}{\textbf{DGL 1. Ordnung}}\\
\hline
	Art & Form & Lösung\\
	\hline
	Separation&
	$y' = f(x)\cdot g(y)$&
	$\dfrac{y'}{g(y)} = f(x)$, nun ist die DGL beidseitig nach x integrierbar\newline
	($dy = y'(x) dx$): $\qquad\int \dfrac{1}{g(y)} dy = \int f(x) dx$ \\
\hline
	separierte Lösung&
	$y'=ax+by+c=z$&
	1. Substitution: $z=ax+by+c\qquad \qquad z'=a+by' =a+b\cdot z$\newline
	$\int\limits_{z_0}^{z}\frac{z'}{a+bz}d\tilde{x} \newline
	[d\tilde{z} = \underbrace{(a+by')}_{z'} d\tilde{x}]$ oder \\
\hline
	Gleichgradigkeit&
	$y'=f\left(\dfrac{y}{x}\right)$ &
	1. Substitution: $z=\dfrac{y}{x} \qquad
	zx=y \quad z'x +z = y' \quad $\newline 
	Glg mit z,x: eine seite z, andere x $\Rightarrow \int$ \\
\hline
	Allgemeine DGL \newline 1. Ordnung&
	$y'+f(x)y = g(x)$\newline
	$g(x)$ : Störterm&
	$ y=e^{-\int f(x) dx}(k+\int g(x)e^{\int f(x)dx}dx) \qquad (k\in\mathbf{R})$\newline
	$Y=y_H+y_p \quad $ Var. K ist Konstante\\
	\hline

\end{tabularx}	
\end{center}
\end{table}	


% % % % % % % % % % % % % % % % % %
%neue Seite
% % % % % % % % % % % % % % % % % %	

\begin{table}[h!]
\begin{center}

% % % % % % % % % % % % % % % % % %
%DGL 1. Ordnung
% % % % % % % % % % % % % % % % % %	
\begin{tabularx}{540pt}{|p{100pt}|X|}
\hline
\rowcolor{Gray}
\multicolumn{2}{|c|}{\textbf{DGL 1. Ordnung}}\\
\hline
	Orthagonaltrajektorien&
	Orthogonaltrajektorien sind die normalen der DGL. Sie stehen senkrecht auf den Kurven die durch die DGL entstehen. \newline
	Um Orthagonaltrajektorien zu erhalten $y'$ ersetzen durch $-\dfrac{1}{y'}$ und DGL lösen.\newline
	$y$ Ableiten $\Rightarrow$ $y' \rightarrow -\dfrac{1}{y'} \rightarrow$ DGL lösen \\
\hline


\end{tabularx}


% % % % % % % % % % % % % % % % % %
%DGL 2. Ordnung
% % % % % % % % % % % % % % % % % %	
\begin{tabularx}{540pt}{|p{120pt}|X|}
\hline
\rowcolor{Gray}
\multicolumn{2}{|c|}{\textbf{DGL 2. Ordnung}}\\
\hline
	Form  & Lösung\\
\hline
	$y''+a_1\cdot y'+a_0\cdot y=g(x)$&
	Wie bei 1. Ordnung: $Y=y_H+y_p$ \newline
	Homogene DGL: $g(x)=0$ \qquad Inhomogene DGL: $g(x)\neq 0$\\
\hline
	\rowcolor{LightCyan}
	\multicolumn{2}{|c|}{Homogene DGL $\qquad y''+a_1\cdot y'+a_0\cdot y=0$}\\
	\multicolumn{2}{|c|}{Charakt. Polynom:
	$\qquad\underline{\lambda^2+a_1\cdot\lambda+a_0=0} \qquad$ von
	$\qquad\underline{y''+a_1\cdot y'+a_0\cdot y=0}$ 
	$\qquad(\lambda_{1,2} = -\frac{a_1}{2} \pm \frac{\sqrt{a_1^2 - 4a_0}}{2})$}
	\\

	\multicolumn{2}{|l|}{$(D > 0)\qquad$Falls:$\lambda_1\neq \lambda_2$ und $\lambda_{1,2} \in R\qquad$: 
	$Y_H=Ae^{\lambda_1x}+Be^{\lambda_2x\qquad}$  
	$\rbrace$ starke Dämpfung}\\
	
	\multicolumn{2}{|l|}{$(D = 0)\qquad$
	Falls: $\lambda_1=\lambda_2$ und $\lambda_{1,2} \in R\qquad$: 
	$Y_H=e^{\lambda_1x}(A+B\cdot x)\qquad$ 
	$\rbrace$ aperiodischer Grenzfall}\\
	
	\multicolumn{2}{|l|}{$(D < 0)\qquad$ 
	Falls $\lambda_{1,2}=-\frac{a_1}{2}\pm \alpha\qquad$: 
	$Y_H=e^{-\frac{1}{2}a_1x}(Acos(\alpha x) +Bsin(\alpha x))$ 
	$\rbrace$ schwache Dämpfung / Schwingfall} \\
	\hline
	Eigenfrequenz&
	$\omega=\alpha=\dfrac{\sqrt{|a_1^2 - 4a_0|}}{2}$\\
\hline	
	\rowcolor{LightCyan}
	\multicolumn{2}{|c|}{inhomogene DGL $\qquad y''+a_1\cdot y'+a_0\cdot y=g(x)$ }\\
	Grundlöseverfahren&
	1. Homogene DGL lösen: $g(x)=0$ setzen ergibt $Y_H$\newline
	2. Anfangsbedingungen in Hom. DGL einsetzen: Wenn möglich $x_0=0\quad \newline 
	y(x_0)=0 \quad y'(x_0)=1 \qquad$ A,B bestimmen\newline
	3. Einsetzen der Hom. Glg. in Faltungsintegral \newline
	4. $y_P(x)=\int\limits_{x_o}^{x} y_H(x+x_0-t)\cdot g(t)dt$\newline
	5. $Y=y_H+y_P$\\
\hline
\end{tabularx}
% % % % % % % % % % % % % % % % % %
%Ansatz in Form des Störgliedes
% % % % % % % % % % % % % % % % % %	
\renewcommand{\arraystretch}{1.1}
\begin{tabularx}{540pt}{|p{270pt}|X|}
	\multicolumn{2}{|c|}{Ansatz in Form des Störgliedes}\\
	\hline 	$\mathbf{f(x)=p_n(x)}$ & 
		($p_n(x)$ und $q_n(x)$ sind Polynome vom gleichen Grad)\\

	 \hline	Fall a: $a_0\neq 0$:          & $y_P = q_n(x)$\\
		Fall b: $a_0 = 0 , a_1\neq 0$:& $y_P=x\cdot q_n(x)$\\
		Fall c: $a_0=a_1=0$:          & $y_P=x^2\cdot q_n(x)$\\
		$a_0$ und $a_1$ beziehen sich auf die \textbf{linke Seite} der DGL & \\
	\hline
	\hline
 		$\mathbf{f(x)=e^{bx}\cdot p_n(x)}$ & \\
	\hline	Fall a: $b$ nicht Nullstelle des char. Polynoms:    &
		$y_P=e^{bx}\cdot q_n(x)$\\
		Fall b: $b$ einfache Nullstelle des char. Polynoms: &
		$y_P=e^{bx}\cdot x \cdot q_n(x)$\\
		Fall c: $b$ zweifache Nullstelle des char. Polynoms:&
		$y_P=e^{bx}\cdot x^2\cdot q_n(x)$\\
	\hline
	\hline
		$\mathbf{f(x) = e^{\alpha x}(p_n(x)\cos \beta x + q_n(x)\sin \beta x)}$ & \\
	 \hline	Fall a: $\alpha + j\beta$ \textbf{nicht Lösung} der charakteristischen Gleichung: &
		$y_p = e^{\alpha x}(r_n(x)\cos \beta x + s_n(x)\sin \beta x)$ \\
		Fall b: $\alpha + j\beta$ \textbf{Lösung} der charakteristischen Gleichung: &
		$y_p = e^{\alpha x} \textbf{x}(r_n(x) \cos \beta x + s_n(x) \sin \beta x)$\\
	\hline
\end{tabularx}
\renewcommand{\arraystretch}{2}
\end{center}
\end{table}	


\begin{table}[h!]
\begin{center}

% % % % % % % % % % % % % % % % % %
%Vorgehen bei einer DGL in From des Störgliedes
% % % % % % % % % % % % % % % % % %	
\begin{tabularx}{540pt}{|p{25pt}|X|}
\hline
\multicolumn{2}{|c|}{\textbf{Vorgehen bei einer DGL in Form des Störgliedes}}\\
\hline
 	1. &$Y_H$ mit $\lambda_1$ und $\lambda_2$ berechnen\\
	2. & Ordnung $n$ anhand der r.h.s der DGL bestimmen
			Koeffizient $b$ anhand der r.h.s der DGL bestimmen\newline
			(Achtung kann aus mehreren Elementen bestehen z.B. $x^2e^x + x$; Superposition)\\
		
	3. &	Anhand der Störglied Tabellen $y_p$ bestimmen\\
	4. & 	$q_n = ax^n + bx^{n-1} + \dots + cx + d$\\
	5. & 	$y_p$ ableiten und in die \textbf{ l.h.s} der DGL einsetzen. $\qquad y_p'' + a_1 y_p' + a_0y_p = f(x)$\\
	6. & 	Koeffizienten bestimmen: $\textcolor{red}{x^2e^x}\cdot 18a + xe^x(6a + 12b) + e^x(2b + 6c) = \textcolor{red}{x^2e^x}$\newline
			\begin{tabular}{ll}
				$18a = 1$ & $18a$ kommt 1mal in der r.h.s vor\\
				$(6a + 12b) = 0$ & $(6a + 12b)$ kommt 0mal vor auf der r.h.s\\
				$(2b + 6c) = 0$ & $(2b + 6c)$ kommt 0mal vor auf der r.h.s
			\end{tabular}\\
	7. & 	Koeffizienten in $y_p$ einsetzen\\
	8. & 	Wenn das Störglied $f(x)$ aus mehreren Teilen besteht (z.B. $x^2e^x + x$), Störglied auseinander nehmen und in zwei Teile $x^2e^x$ und $x$ unterteilen und Schritt 3 - 6 wiederholen\\
	9. & 	$y = Y_H + y_{p1} + y_{p2} + \dots$\\

\hline
\end{tabularx}
\begin{tabularx}{540pt}{|p{120pt}|X|}
Superpositionsprinzip & $f(x)=c_1f_1(x)+c_2f_2(x)$\newline
$y_1$ ist spezielle Lösung der DGL $\qquad$
$y_1''+a_1\cdot y_1'+a_0\cdot y_1=c_1f_1(x)$ \newline
$y_2$ ist spezielle Lösung der DGL $\qquad$
$y_2''+a_1\cdot y_2'+a_0\cdot y_2=c_2f_2(x)$ \newline
dann ist $y_P=c_1y_1+c_2y_2$\\
\hline
\end{tabularx}


% % % % % % % % % % % % % % % % % %
%Lineare DGL n. Ordnung mit konstanten Koeffizienten
% % % % % % % % % % % % % % % % % %	
\begin{tabularx}{540pt}{|p{120pt}|X|}
\hline
\rowcolor{Gray}
\multicolumn{2}{|c|}{\textbf{Lineare DGL n. Ordnung mit konstanten Koeffizienten}}\\
\hline
	Form & $\sum\limits_{k=0}^na_ky^{(k)}= y^{(n)}+a_{n-1}\cdot y^{(n-1)}+\ldots +a_0\cdot y=g(x)$\\
\hline
\end{tabularx}
\renewcommand{\arraystretch}{1}
\begin{tabularx}{540pt}{|p{130pt}p{240pt}X|}
\multicolumn{3}{|c|}{n-verschiedene Homogene Lösungen}\\
Fall a: r reelle Lösungen \newline$\lambda_1$: 
	& $y_1=e^{\lambda_1x}$, $y_2=xe^{\lambda_1x}$, \ldots
	,$y_r=x^{r-1}e^{\lambda_1x}$ 
	& Starke Dämpfung / Kriechfall\\
Fall b: $k$ komplexe Lösungen\newline $\lambda_2=\alpha +j\beta$: 
	&$y_1=e^{\alpha x}\cos(\beta x)$, \ldots, $y_k=e^{\alpha x}x^{k-1}\cos(\beta
x)$\newline
$y_{k+1}=e^{\alpha x}\sin(\beta x)$, \ldots, $y_{2k}=e^{\alpha
x}x^{k-1}\sin(\beta x)$
	& Schwache Dämpfung /\newline Schwingfall\\

	\multicolumn{3}{|l|}{$Y_H = Ay_1 + By_2 + Cy_3 + ... + Ny_n$}\\
\hline
\end{tabularx}
\renewcommand{\arraystretch}{2}
\end{center}
\end{table}	


\begin{table}[h!]
\begin{center}

% % % % % % % % % % % % % % % % % %
%Allgemeinste Lösung des partikulären Teils
% % % % % % % % % % % % % % % % % %	
\begin{tabularx}{540pt}{|p{300pt}|X|}
\hline
	\multicolumn{2}{|c|}{Allgemeinste Lösung des partikulären Teils}\\
\hline
	\multicolumn{2}{|c|}{$\underbrace{\sum_{k=0}^n a_k y^{(k)}}_{f(y,y',y'',\ldots)} = \underbrace{e^{\alpha x} (p_{m1}(x) \cos (\beta x) + q_{m2}(x) \sin (\beta x))}_{\text{Störglied}} \qquad \lambda \text{ aus Homogenlösung}$}\\
\hline
	Unterscheide die Lösungen des charakteristischen Polynoms ($\lambda$): &
	mit m = max(m1, m2)\\

	Fall a: $\alpha + j\beta \neq \lambda$, so ist &
	$y_P = e^{\alpha x}(r_m(x)\cos(\beta x) + s_m(x) \sin(\beta x))$\\
	Fall b: $\alpha + j\beta$  ist u-fache Lösung von $\lambda$, so ist &
	$y_P = e^{\alpha x} x^u (r_m(x) \cos(\beta x) + s_m(x) \sin(\beta x))$\newline
	u-fache Resonanz\\
\hline
\end{tabularx}
% % % % % % % % % % % % % % % % % %
%Grundlöseverfahren
% % % % % % % % % % % % % % % % % %	
\begin{tabularx}{540pt}{|p{350pt}|X|}
\hline
	\multicolumn{2}{|c|}{Grundlöseverfahren}\\
\hline
	$\begin{pmatrix}
	g(x_0)=  & 0 & = & Ay_1(x_0)+By_2(x_0)+\ldots +Ny_n(x_0)\\
	g'(x_0)= & 0 & = & Ay_1'(x_0)+By_2'(x_0)+\ldots +Ny_n'(x_0)\\
	\vdots  & \vdots & \\                            
	g^{(n-1)}(x_0)= & 1 & = & Ay_1^{(n-1)}(x_0)+By_2^{(n-1)}(x_0)+\ldots
	+Ny_n^{(n-1)}(x_0)
	\end{pmatrix}$ &
	
	ergibt $c_1,\ldots ,c_n$ für\newline
	$y_{P}(x)=\int\limits_{x_0}^x{g(x+x_0-t)f(t)dt}$\\
\hline
\end{tabularx}

\begin{tabularx}{540pt}{|p{120pt}|X|}
	Anfangswertproblem&
	$y(x_0) = y_0 \qquad y'(x_0) = y_1 \qquad y''(x_0) = y_2 \qquad \dots \qquad y^{(n-1)}(x_0) = y_{n-1}$\\
\hline
\end{tabularx}
% % % % % % % % % % % % % % % % % %
%Lineare Differentialgleichungssysteme erster Ordnung mit konstanten Koeffizienten
% % % % % % % % % % % % % % % % % %	

\begin{tabularx}{540pt}{|p{8cm}X|}
\hline
\multicolumn{2}{|c|}{\textbf{Lineare Differentialgleichungssysteme erster Ordnung mit konstanten Koeffizienten}}\\
\hline
	\textbf{Form:}& $	\begin{matrix} \dot{x}=ax+by+f(t) \\ \dot{y}=cx+dy+g(t) \end{matrix} = \left(\begin{matrix} \dot{x} \\ \dot{y} \end{matrix}\right) = 
				\left(\begin{matrix} a & b \\ c & d \end{matrix}\right) \left(\begin{matrix} x \\ y \end{matrix}\right) + \left(\begin{matrix} f(t) \\ g(t) \end{matrix}\right)$ \\


	\textbf{Die allgem. Lösung ergibt sich aus der DGL:}&
	$\underbrace{\ddot{x}-(a+d)\dot{x}+(ad-bc)x=\dot{f}(t)-df(t)+bg(t)}_{\text{normale DGL 2.Ordnung} \rightarrow \text{nach $x$ auflösen}}$\\
	& $y=\frac{1}{b}(\dot{x}-ax-f(t)))$\\

\textbf{Anfangsbedinung:} &
$x_0(t_0) = x_0, \dot{x}_0(t_0) = ax_0 + by_0 + f(t_0)$\\
\hline
\multicolumn{2}{|c|}{\textbf{Anordnung beachten!} Gesuchte Grösse immer zu oberst (in diesem Fall ist die gesuchte Grösse $x$)}\\
\hline
\end{tabularx} 



\end{center}
\end{table}	

